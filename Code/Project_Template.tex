\PassOptionsToPackage{unicode=true}{hyperref} % options for packages loaded elsewhere
\PassOptionsToPackage{hyphens}{url}
%
\documentclass[12pt,]{article}
\usepackage{lmodern}
\usepackage{amssymb,amsmath}
\usepackage{ifxetex,ifluatex}
\usepackage{fixltx2e} % provides \textsubscript
\ifnum 0\ifxetex 1\fi\ifluatex 1\fi=0 % if pdftex
  \usepackage[T1]{fontenc}
  \usepackage[utf8]{inputenc}
  \usepackage{textcomp} % provides euro and other symbols
\else % if luatex or xelatex
  \usepackage{unicode-math}
  \defaultfontfeatures{Ligatures=TeX,Scale=MatchLowercase}
    \setmainfont[]{Times New Roman}
\fi
% use upquote if available, for straight quotes in verbatim environments
\IfFileExists{upquote.sty}{\usepackage{upquote}}{}
% use microtype if available
\IfFileExists{microtype.sty}{%
\usepackage[]{microtype}
\UseMicrotypeSet[protrusion]{basicmath} % disable protrusion for tt fonts
}{}
\IfFileExists{parskip.sty}{%
\usepackage{parskip}
}{% else
\setlength{\parindent}{0pt}
\setlength{\parskip}{6pt plus 2pt minus 1pt}
}
\usepackage{hyperref}
\hypersetup{
            pdftitle={Analysis of drinking water water contaminant occurrence in the Northeast and Southeast United States},
            pdfauthor={Rachel Gonsenhauser},
            pdfborder={0 0 0},
            breaklinks=true}
\urlstyle{same}  % don't use monospace font for urls
\usepackage[margin=2.54cm]{geometry}
\usepackage{graphicx,grffile}
\makeatletter
\def\maxwidth{\ifdim\Gin@nat@width>\linewidth\linewidth\else\Gin@nat@width\fi}
\def\maxheight{\ifdim\Gin@nat@height>\textheight\textheight\else\Gin@nat@height\fi}
\makeatother
% Scale images if necessary, so that they will not overflow the page
% margins by default, and it is still possible to overwrite the defaults
% using explicit options in \includegraphics[width, height, ...]{}
\setkeys{Gin}{width=\maxwidth,height=\maxheight,keepaspectratio}
\setlength{\emergencystretch}{3em}  % prevent overfull lines
\providecommand{\tightlist}{%
  \setlength{\itemsep}{0pt}\setlength{\parskip}{0pt}}
\setcounter{secnumdepth}{5}
% Redefines (sub)paragraphs to behave more like sections
\ifx\paragraph\undefined\else
\let\oldparagraph\paragraph
\renewcommand{\paragraph}[1]{\oldparagraph{#1}\mbox{}}
\fi
\ifx\subparagraph\undefined\else
\let\oldsubparagraph\subparagraph
\renewcommand{\subparagraph}[1]{\oldsubparagraph{#1}\mbox{}}
\fi

% set default figure placement to htbp
\makeatletter
\def\fps@figure{htbp}
\makeatother

\usepackage{etoolbox}
\makeatletter
\providecommand{\subtitle}[1]{% add subtitle to \maketitle
  \apptocmd{\@title}{\par {\large #1 \par}}{}{}
}
\makeatother

\title{Analysis of drinking water water contaminant occurrence in the Northeast
and Southeast United States}
\providecommand{\subtitle}[1]{}
\subtitle{\url{https://github.com/rachel-gonsenhauser/Final_Project_Environmental_Data_Analytics}}
\author{Rachel Gonsenhauser}
\date{}

\begin{document}
\maketitle

\newpage
\abstract

\begin{quote}
\end{quote}

\newpage
\tableofcontents 
\newpage
\listoftables 
\newpage
\listoffigures 
\newpage

\hypertarget{rationale-and-research-questions}{%
\section{Rationale and Research
Questions}\label{rationale-and-research-questions}}

\begin{quote}
While the EPA establishes standards for 90 drinking water contaminants
by means of the federal Safe Drinking Water Act (SDWA) and its
regulations, public water systems still often struggle to remain in
compliance with such policies (USEPA, 2020). This issue of compliance
with the SDWA can stem from myriad causes, for instance financial
capacity of the water system. This is especially concerning in areas
where geologic conditions and/or anthropogenic activities frequently
introduce contaminants into drinking water supplies. Additionally, some
known contaminants still have yet to be regulated by the EPA, such as
poly- and perfluoroaklyl substances (PFAS), which introduces even more
complexity to the issue of water quality monitoring of drinking water
sources.
\end{quote}

\begin{quote}
This analysis seeks to investigate the co-occurrence of water quality
indicators including arsenic, trihalomethane, and uranium, and PFAS,
which originate from both geogenic and anthropogenic sources.
Additionally, given pervasive questions related to environmental justice
and how socioeconomic factors may be related to water quality
indicators, this analysis seeks to examine trends between water quality
indicators and county-level median household income (MHI) and size of
the population served by a given community water system (CWS), which is
often a proxy for how rural an area is and the financial capacity of a
CWS. Additionally, questions regarding how contaminant occurrence
differs across time and between states are explored.
\end{quote}

\begin{quote}
To narrow the scope of this project, most analyses are targeted to
southeastern region states and northeastern region states. Additionally,
individual case studies of Massachusetts and North Carolina are explored
in further depth. Due to issues of PFAS data limitations, discussed in
more detail in the subsequent section, analyses using PFAS data are
limited.
\end{quote}

\begin{quote}
The dataset used for this analysis a processed dataset which combined
individual drinking water contaminant and MHI data available from the
Centers for Disease Control adn Prevention (CDC)'s National
Environmental Public Health Tracking Network.
\end{quote}

\newpage

\hypertarget{dataset-information}{%
\section{Dataset Information}\label{dataset-information}}

\newpage

\hypertarget{exploratory-analysis}{%
\section{Exploratory Analysis}\label{exploratory-analysis}}

\newpage

\hypertarget{analysis}{%
\section{Analysis}\label{analysis}}

\hypertarget{question-1-insert-specific-question-here-and-add-additional-subsections-for-additional-questions-below-if-needed}{%
\subsection{Question 1: \textless{}insert specific question here and add
additional subsections for additional questions below, if
needed\textgreater{}}\label{question-1-insert-specific-question-here-and-add-additional-subsections-for-additional-questions-below-if-needed}}

\hypertarget{question-2}{%
\subsection{Question 2:}\label{question-2}}

\newpage

\hypertarget{summary-and-conclusions}{%
\section{Summary and Conclusions}\label{summary-and-conclusions}}

\newpage

\hypertarget{references}{%
\section{References}\label{references}}

\begin{quote}
United States Environmental Protection Agency (USEPA). 2020. Safe
Drinking Water Act (SDWA). Retrieved from:
\url{https://www.epa.gov/sdwa}.
\end{quote}

\end{document}
